\documentclass[
%journal=ancac3, % for ACS Nano
%journal=acbcct, % for ACS Chem. Biol.
journal=jacsat, % for undefined journal
manuscript=article]{achemso}

\usepackage[version=3]{mhchem} % Formula subscripts using \ce{}

\newcommand*{\mycommand}[1]{\texttt{\emph{#1}}}

\author{Andrew N. Other}
\author{Fred T. Secondauthor}
\altaffiliation{Current address: Some other place, Othert\"own,
Germany}
\author{I. Ken Groupleader}
\email{i.k.groupleader@unknown.uu}
\affiliation[Unknown University]
{Department of Chemistry, Unknown University, Unknown Town}
\author{Susanne K. Laborator}
\email{s.k.laborator@bigpharma.co}
\affiliation[BigPharma]
{Lead Discovery, BigPharma, Big Town, USA}
\author{Kay T. Finally}
\affiliation[Unknown University]
{Department of Chemistry, Unknown University, Unknown Town}


\title[\texttt{achemso} demonstration]
{Smart City Scenarios}

\begin{document}

\begin{abstract}

\end{abstract}

\section{Introduction}

\section{Scenarios}

In this section we describe smart city scenarios that can be simulated using our smart city simulator.

\begin{itemize}
\item \textbf{Traffic:} It is possible to simulate many traffic scenarios, such as the cause and effects of traffic jams, the best route of public transport, ant the optimal number of buses. The main actors of those scenarios are vehicles, people, semaphores, and sensors.
Detailed Scenarios:
(Management, Jams, Public Transport, Bike, Parking,...)

\item \textbf{Water and Energy Management:} In a smart city, it is possible to manage and control the use of city resources. It is possible to measure the amount of those resources used in buildings, streets and neighborhoods. It is also possible to measure the quality of those services. The smart city simulator can simulate many scenarios, such as, a water quality measurement with sensors, the amount of water or energy used by smart buildings, and changes in the usage of water or energy. The main actors of those scenarios are water reservoirs, energy stations, smart buildings, and sensors.


\item \textbf{Health Care Scenarios:} Many authors cite health care scenarios in smart cities. Some examples of scenarios that can be simulated are: Ambulance having to cross the city, communicating with other vehicles or semaphores, devices that send information about patients to doctors, and can start an activity, such as  calling an ambulance.
Detailed Scenarios:

\item \textbf{Disaster management:} One important IoT and Smart City application is the preventing and management of disaster, such as fire, 

\item \textbf{City Measurements:} One of the most cited scenarios of smart cities is the deploying of a sensor network in the city. With this network is possible to collect a lot of city information such as air pollution, luminosity, water pollution, and noise. Some of the possibibilities of simulation in this scenario is to simulate the cost of deploying such sensor network, the minimum amount of sensor necessary, and the volume of data generated by those sensors.
Detailed Scenarios:

\item \textbf{Waste Management:} The management of waste services in the city is also cited by many authors, some examples of scenarios in this area are the using of a sensor network to notify when trash cans are full, the visualization and control of trash vehicles, and the prediction of when waste disposals will be full with the growth of population.

\item \textbf{Security:} (Crime reports, light posts, cameras)

\end{itemize}


\section{Actors and Events}

This section presents the main actors and events necessaries to implement the simulations of the presented scenarios.


\begin{itemize}

\item \textbf{Sensors and Actuators:} Actors to model the IoT network in the city. Those actors are important to model many scenarios, such as city monitoring, waste management, and disater management. The sensors  can be positioned in the city, and can generate the following events:
	- GenerateData: Generate some type of data, such as temperature, number of 	cars passing in a street, and luminosity.
	- Failure: The failure of a sensor, and it impacts in a sensor network.
	- Add and Remove	: The addition or a removal of a sensor and the impacts in 	the network.

\item \textbf{Vehicles:} Actors to model the traffic and traffic behavior of the city. This can be used mainly in the traffic management scenarios, but can be also used in disaster management, and in security scenarios. It is possible to implement specific type of vehicles, such as cars, buses, ambulances, trucks, and bikes.
	- Move: Move the vehicle through the city.
	- Accident: Simulate an accident and it consequences.
	- Start/Stop: Start or stopa new actor from other event or actor, such as a building.
	- Load/Unload (Waste Trucks): Load/Unload waste trucks, the waste trucks can be loaded with trash from buildings and unload the trash in waste disposals.
	- Collect Data: Collect context-aware data from vehicles, such as  GPS position, speed, travel time.
\item \textbf{People:} Actor to model the people of the city and their devices. It can be important to simulate many scenarios, such as disaster scenarios, big events scenarios, the use of users devices.
	- GetData: Retrieve data and context-aware data from user devices.
	- Move: 
	- Start/Stop
	
\item \textbf{Systems:} Systems that receive information of other actors and can start an event, such as emergence system and health care systems.
	- Receive Data:
	
\item \textbf{Semaphore:} To model the traffic system of the city.
	- Open/Close
	- Failure
	- Add and Remove
	
\item \textbf{Buildings:} (Houses, Offices, Waste Disposals, Hospitals, Water Reservoirs,...): To model smart building that sends informations to people or system, such as water and energy consumption, emergence alerts. And it can be used to generate other actors, such as vehicles and people.
	- Receive Data: Receive data from others actors, such as waste disposals that receives waste from vehicles, houses that start cars, and fire department that sends cars to a detected disaster.
	- Start other actors (Vehicles, People)
	- GetData (Energy and water consumption, trash)


\end{itemize}

\section{Communities}



\section{Metrics}


\section{Examples of questions that the simulator have to answer}

1. How a block in a street can change the behavior of the traffic system?

2. What is the effect of a failure in a (or in a group of) semaphore?

3. What is the difference in the transit using a smart semaphore system?

4. What a change in the water consumption can save water of the reservoirs?

5. What is the number of sensors to measure the quality of the water system of 
a city?

6. What are the number of sensors that is necessary to measure the air 
pollution in the whole city?

7. How many sensors are necessary to identify emergencies (flood, fire, 
landslide) in the city?

8. How many sensors are necessary to identify available parking spaces?

9. What is the cost of the sensors of the question 5, 6, 7, and 8?

10. A change in the traffic system can decrease the CO2 emission?

11. The sensor network of the question 6 is able to detect this decrease?

12. It is better to move across the city by public transport, by car or by taxi?

\section{Other tests}


1. Simulate different network protocols.

2. Simulate the growth of the usage of the bike paths. (Simulate better bike paths)

3. Simulate sensors to detect the number of cyclists.

4. Simulate an ambulance using communication with sensors and semaphores.

5. Detection of rubbish levels in containers to optimize the trash collection routes

6. Sound monitoring in bar areas and central zones in real-time

\section{Simulator Requirements}
\begin{itemize}

\item \textbf{Scalability:} As a smart city scenarios can be very large, with millions of cars, sensors, people, and other actors, it is important to the simulator support the growth in the number of Nodes and Events in the simulation. 
\item \textbf{Synchronization:} 
\item \textbf{Extensibility:}
\item \textbf{Generate Real-Time Data and Reports:} It is important to generate real-time data to provide data to 
\item \textbf{Integration with a map platform:} It is important to simulations that is important to visualize the simulation results, such as traffic scenarios.

\end{itemize}

\section{Simulators}

\section{Conclusions}

\bibliography{sample}

\end{document}

